\documentclass{jsarticle}			% 日本語用
\usepackage{listings}				% ソースコード用
\usepackage[dvipdfmx]{graphicx}  	% 図の挿入用
\usepackage{amsmath,amssymb}		% 数式
\usepackage{float}					% 図・表位置[H]
\usepackage{subcaption}				% subtable可
\usepackage{hyperref}				% URL

\makeatletter
	\renewcommand*\l@subsection{\@dottedtocline{2}{10em}{5em}}
\makeatother
		% https://tex.stackexchange.com/questions/33841/how-to-modify-the-space-between-the-numbers-and-text-of-sectioning-titles-in-the/33842
		%The command \@dottedtocline expects the following parameters:
		%\renewcommand{\l@<typ>}{\@dottedtocline{<level>}%
		%                                       {<indentation>}%
		%                                       {<numwidth>}}

\title{タイトル}
\author{pollenJP}

\begin{document}
	
	\maketitle
	
	\renewcommand\contentsname{目次(もくじだよ)}
	\tableofcontents
	\addtocontents{toc}{
		{\fontsize{20.0pt}{0pt}\selectfont 目次}
		\hfill\textbf{Page}\par}
	\addcontentsline{toc}{section}{Preface}
	\addcontentsline{toc}{subsection}{Preface}
	
	\section{TableOfContents}
		\begin{itemize}
			\item \url{https://tex.stackexchange.com/questions
				/33841/how-to-modify-the-space-between-the-numbers-
				and-text-of-sectioning-titles-in-the/33842
				}
			\item \url{https://en.wikibooks.org/wiki/LaTeX/Document_Structure}
			\item \url{https://cmry.github.io/notes/latextoc}
		\end{itemize}
	
	\section{Who is pollenJP?}
		\subsection{pollenJPJP}
			\subsubsection{pollenJPJPJP}
				\begin{enumerate}
					\item aaa
					\item bbb
					\item ccc
					\item ddd
				\end{enumerate}
				%
				\begin{description}
					\item[Block のパラメータの設定 Gaussian Noise Generator]\mbox{}
						\begin{itemize}
							\item 平均値(Mean Value)=0
							\item 分散(Variance)=0.8
							\item サンプル時間=0.01 秒
						\end{itemize}
					\item[シミュレーション時間]\mbox{}
						\begin{itemize}
							\item 終了時間 10 秒
						\end{itemize}
					\item[その他]\mbox{}
						\begin{itemize}
							\item Signal To Workspace,Display は,設定の変更は必要ない.
						\end{itemize}
					\item[hello]\mbox{}\\
						あいうえお
				\end{description}

			\subsubsection{pollenJPJPJP2}
				hello
		\subsection{pollenJPJP2}
			hello
	
	\section{表}
		以下の表\ref{課題1.2: M=2の表}を示す.
		% table
		\begin{table}[H]  % t(top),b(bottom),p(page),h(できればhere),H(必ずhere)
			\caption[M=2]{課題1.2: M=2の表}
			\label{課題1.2: M=2の表}
			\begin{center}
				\setlength{\tabcolsep}{3pt}
				\footnotesize
				\begin{tabular}{|c|c|c} \hline
					シンボル & ビット \\ \hline \hline
					0 & 0 \\ \hline
					1 & 1 \\ \hline
				\end{tabular}
			\end{center}
		\end{table}
		
		\subsection{subtable}
			以下の表\ref{課題1.2: 課題}
			% table
			\begin{table}[H]
				\begin{subtable}{0.49\columnwidth}
					\begin{center}
						\caption[M=2]{M=2の表}
						\label{課題1.2: M=2の表}
						\setlength{\tabcolsep}{3pt}
						\footnotesize
						\begin{tabular}{|c|c|c}
							\hline
							シンボル & ビット \\ \hline \hline
							0 & 0 \\ \hline
							1 & 1 \\ \hline
						\end{tabular}
					\end{center}
				\end{subtable}
				\begin{subtable}{0.49\columnwidth}
					\begin{center}
						\caption[M=4]{M=4の表}
						\label{課題1.2: M=4の表}
						\setlength{\tabcolsep}{3pt}
						\footnotesize
						\begin{tabular}{|c|c|c}
							\hline
							シンボル & ビット \\ \hline \hline
							0 & 00 \\ \hline
							1 & 01 \\ \hline
							2 & 10 \\ \hline
							3 & 11 \\ \hline
						\end{tabular}
					\end{center}
				\end{subtable}
				\begin{subtable}{\columnwidth}
					\begin{center}
						\caption[M]{課題}
						\label{課題1.2: 課題}
						\setlength{\tabcolsep}{3pt}
						\footnotesize
						\begin{tabular}{|l|c|c|c|c|c|c|c|c|c|c|c|c|}
							\hline
							
								& \multicolumn{6}{|l|}{M=2}
								& \multicolumn{6}{|l|}{M=4} \\ \hline
							時間 
								& 0 & 1 & 2 & 3 & 4 & 5
								& 0 & 1 & 2 & 3 & 4 & 5 \\ \hline
							シンボル列
								& 1 & 0 & 0 & 0 & 1 & 1
								& 2 & 0 & 0 & 1 & 2 & 3 \\ \hline
							ビット列
								&  1 &  0 &  0 &  0 &  1 &  1
								& 10 & 00 & 00 & 01 & 10 & 11 \\ \hline
						\end{tabular}
					\end{center}
				\end{subtable}
			\end{table}

	
	\section{図} \label{section図}
		以下の図
		%figure
		\begin{figure}
		
		\end{figure}
		% figure
%		\begin{figure}[H]  % t(top),b(bottom),p(page),h(できればhere),H(必ずhere)
%			\begin{center}
%				\includegraphics[width=0.8\textwidth]{fig01_kadai1-1.png}
%				\caption{課題1.1:構成}
%				\label{課題1.1:構成}
%			\end{center}
%		\end{figure}
%
	
	% save old section configure
	\let\oldthesection=\thesection
	\let\oldthesubsection=\thesubsection
%	\setcounter{section}{0}
	\renewcommand{\thesection}{課題\arabic{section}}
	\renewcommand{\thesubsection}{課題\arabic{section}.\arabic{subsection}}
	\section{セクションタイトル替え} \label{課題1}
		\subsection*{実験内容}
		\subsection{サブタイトルも変える}
			\subsubsection{subsub}
				\begin{itemize}
					\item \url{
						https://groups.google.com/forum/#!topic/comp.text.tex/EcAPeYr-ySE}
				\end{itemize}
			\subsubsection*{subsub2}
		\subsection{次のセクションでタイトルを戻す}

	\let\thesection=\oldthesection
	\let\thesubsection=\oldthesubsection
	\section{URLを使用}
		\begin{itemize}
			\item \url{https://www.sharelatex.com/learn/Hyperlinks}
		\end{itemize}
		
		


\end{document}

		
		

