\documentclass[dvipdfmx, 12pt, aspectratio=169]{beamer}  %8,9,10,11,12,14,17,20pt
\usepackage{bxdpx-beamer}  % dvipdfmx対応
\usepackage{pxjahyper}  % しおりの和文対応
                        % "hyperref"の日本語文字化け対策
\usepackage{minijs}
\renewcommand{\kanjifamilydefault}{\gtdefault}  % 和文をゴシックに

\usepackage{float}  % 図・表位置[H]

% subfigure
\usepackage{subcaption}

% put block text at the absolute position
% https://nxg.me.uk/dist/textpos/textpos.pdf
% https://tex.stackexchange.com/questions/204196/beamer-changing-textwidth
\usepackage[absolute,overlay]{textpos}

%====================
% url custom  
%====================
\usepackage{hyperref}  % Link/URL
\hypersetup{
    colorlinks,
    linkcolor=,
    filecolor=magenta,
    urlcolor=cyan,
    bookmarksopen,
    bookmarksdepth=2,
    bookmarkstype=toc,  % toc link
%    linktocpage=true,   % tocのページ番号の方にリンクが現れる
    bookmarks=true,     % toc link
    % ref
    % https://texwiki.texjp.org/?hyperref#h026c44c
}

%====================
% 図とcaptionとの間の距離を調整
%====================
% [spacing - How can I modify vertical space between figure and caption? - TeX - LaTeX Stack Exchange](https://tex.stackexchange.com/a/45996/185738)
% [floats - How to remove spacing between figure and caption in the beamer class - TeX - LaTeX Stack Exchange](https://tex.stackexchange.com/a/226154/185738)
%\setlength{\abovecaptionskip}{-5pt plus 0pt minus 0pt}
%\setlength{\belowcaptionskip}{0pt plus 0pt minus 0pt}

%====================
% ハイライト - highlight
%====================
% [A Guide to Using Beamer - University of Windsor 08-02.pdf](http://www1.uwindsor.ca/math/sites/uwindsor.ca.math/files/08-02.pdf)

%====================
% [TikZ — Tasuku Soma's webpage](https://www.opt.mist.i.u-tokyo.ac.jp/~tasuku/tikz.html)
\usepackage{tikz}
\usetikzlibrary{positioning}
\newcommand{\highlight}[2][yellow]{\tikz[baseline=(x.base)]{\node[rectangle,rounded corners,fill=#1!10](x){#2};}}
\newcommand{\highlightcap}[3][yellow]{\tikz[baseline=(x.base)]{\node[rectangle,rounded corners,fill=#1!10](x){#2} node[below of=x, color=#1]{#3};}}
\usepackage{pxpgfmark}   % TikZを読み込む後に書く
\usetikzlibrary{shapes.callouts}

%====================
% テーマの名前
%====================
\usetheme{metropolis}

%===============================================================================
%===============================================================================
%===============================================================================

\begin{document}
  \begin{frame}
    \frametitle{}
    \title{コンピュータ画像処理}
    \subtitle{2.3 画像処理アルゴリズムの形態 (p.25 - p.38)}
    \author{上智大学 川中研 B4 杉崎弘明}
    \date{2019/04/15 Mon \\
    }
    \maketitle
  \end{frame}

  \begin{frame}
    \begin{tikzpicture}
      \node[rectangle,fill=cyan!10,text width=3cm,text centered,
        rounded corners,minimum height=1.5cm]{root};
    \end{tikzpicture}

    \begin{tikzpicture}
        \tikzset{block/.style={rectangle, fill=cyan!10, text width=3cm,
                               text centered, rounded corners,
                               minimum height=1.5cm}};
        \node[block] {root};
    \end{tikzpicture}
  \end{frame}

  \begin{frame}
    \begin{tikzpicture}
      \tikzset{block/.style={rectangle, fill=cyan!10, text width=3cm,
                             text centered, rounded corners,
                             minimum height=1.5cm}};
      \node[block] {root}
        child{ node[block] {child1} }
        child{ node[block] {child2} }
        child{ node[block] {child3} };
    \end{tikzpicture}
  \end{frame}

  \begin{frame}
    \begin{align*}
        f(\mathbf{w}, b) =
            \highlightcap[red]{$\displaystyle \sum_{i=1}^k \left(y_i - \mathbf{w}^\top \mathbf{x}_i - b \right)^2$}{経験誤差}
            +
            \highlightcap[blue]{$\displaystyle \frac{\lambda}{2} \left\|\mathbf{w} \right\|^2$}{正則化項}
    \end{align*}
  \end{frame}

  \begin{frame}
    \begin{textblock}{5.0}(10,8)
      \begin{tikzpicture}
        \tikzset{block/.style={rectangle, fill=cyan!10, text width=3cm,
                               text centered, rounded corners,
                               minimum height=1.5cm}};
        \node[block] {root}
          child{ node[block] {child1} }
          child{ node[block] {child2} }
          child{ node[block] {child3} };
      \end{tikzpicture}
    \end{textblock}

    \begin{tikzpicture}
      %\node[rectangle callout, fill=red!80, white, callout absolute pointer={(bgraph.north)}, above=of bgraph]{2部グラフ};
      \node[rectangle callout, fill=red!80, white, text width=4em,
            callout relative pointer={{(1,1)}}]{2部グラフ\\だよ};
    \end{tikzpicture}

  \end{frame}

\end{document}
